\documentclass{ctexart}

%%代码块包
\usepackage{listings}
\usepackage{fontspec}

%%自定义颜色
\usepackage{xcolor}
\definecolor{codegreen}{rgb}{0,0.6,0}
\definecolor{codegray}{rgb}{0.5,0.5,0.5}
\definecolor{codepurple}{rgb}{0.58,0,0.82}
\definecolor{backcolour}{rgb}{0.95,0.95,0.92}

%%自定义代码块样式
\lstdefinestyle{mystyle}{
    basicstyle=\fontspec{Consolas}\small,
    keywordstyle=\color[rgb]{0.627,0.126,0.941}\fontspec{Consolas Bold}, % 关键字
    commentstyle=\color[rgb]{0.133,0.545,0.133}\fontspec{Consolas Italic}, % 注释
    numberstyle=\small\fontspec{Consolas}, % 行号
    stringstyle=\color{blue},
    aboveskip={1.0\baselineskip},
    belowskip={1.0\baselineskip},
    columns=fixed,
    extendedchars=true,
    breaklines=true,
    tabsize=4,
    prebreak=\raisebox{0ex}[0ex][0ex]{\ensuremath{\hookleftarrow}},
    frame=single, %single
    numbersep=14pt, %-12pt
    showtabs=false,
    showspaces=false,
    showstringspaces=false,
    numbers=left,
    stepnumber=1,
    captionpos=t,
    escapeinside={\%*}{*)},
    % identifierstyle=\color{blue},
    }
\lstset{style=mystyle}

%%修改目录中显示的代码块列表名称
\renewcommand{\lstlistoflistings}{Code} %可以把code修改成其它想要显示的名字

\begin{document}

%% 添加目录
\tableofcontents
\lstlistoflistings%%列出代码块


\begin{lstlisting}[language=Python,caption=python code]
import numpy as np

tp = (1,2,3,4,5,6,7,8,9,10)
print(tp[:5])
print(tp[5:])
"string"

def incmatrix(genl1,genl2):
    m = len(genl1)
    n = len(genl2)
    M = None #to become the incidence matrix
    VT = np.zeros((n*m,1), int)  #dummy variable
    
    #compute the bitwise xor matrix
    M1 = bitxormatrix(genl1)
    M2 = np.triu(bitxormatrix(genl2),1) 

    for i in range(m-1):
        for j in range(i+1, m):
            [r,c] = np.where(M2 == M1[i,j])
            for k in range(len(r)):
                VT[(i)*n + r[k]] = 1;
                VT[(i)*n + c[k]] = 1;
                VT[(j)*n + r[k]] = 1;
                VT[(j)*n + c[k]] = 1;
                
                if M is None:
                    M = np.copy(VT)
                else:
                    M = np.concatenate((M, VT), 1)
                
                VT = np.zeros((n*m,1), int)
    
    return M
\end{lstlisting}

\begin{lstlisting}[language=Python,caption=python code]
    import numpy as np
        
    def incmatrix(genl1,genl2):
        m = len(genl1)
        n = len(genl2)
        M = None #to become the incidence matrix
        VT = np.zeros((n*m,1), int)  #dummy variable
        
        #compute the bitwise xor matrix
        M1 = bitxormatrix(genl1)
        M2 = np.triu(bitxormatrix(genl2),1) 
    
        for i in range(m-1):
            for j in range(i+1, m):
                [r,c] = np.where(M2 == M1[i,j])
                for k in range(len(r)):
                    VT[(i)*n + r[k]] = 1;
                    VT[(i)*n + c[k]] = 1;
                    VT[(j)*n + r[k]] = 1;
                    VT[(j)*n + c[k]] = 1;
                    
                    if M is None:
                        M = np.copy(VT)
                    else:
                        M = np.concatenate((M, VT), 1)
                    
                    VT = np.zeros((n*m,1), int)
        
        return M
\end{lstlisting}


\end{document}