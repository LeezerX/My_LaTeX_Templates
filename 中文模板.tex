%%设置文档格式,将onecolumn换成twocolumn即为双栏排版
\documentclass[11pt,onecolumn,a4paper]{ctexart}

% 打数学公式并且让他更好看
\usepackage{amsmath}

%%设置图片
\usepackage{graphicx}
\usepackage{float}
\usepackage[export]{adjustbox}
\usepackage{subfigure}

%%设置纸张格式和页边距
\usepackage{geometry}
\geometry{a4paper,left=2.54cm,right=2.54cm,top=3.09cm,bottom=3.09cm}

%%设置行间距
\linespread{1.2}

%%为图片和表格自动编号,name=后的东西即为标头
\usepackage{caption}
\captionsetup[figure]{name=}
\captionsetup[table]{name=}

%%图片目录
\graphicspath{{./figures/}}

%%在文档中改变字体的颜色
\usepackage{xcolor}
\newcommand{\red}[1]{\textcolor[rgb]{1.00,0.00,0.00}{#1}}
\newcommand{\blue}[1]{\textcolor[rgb]{0.00,0.00,1.00}{#1}}
\newcommand{\green}[1]{\textcolor[rgb]{0.00,1.00,0.00}{#1}}
\newcommand{\darkblue}[1]
{\textcolor[rgb]{0.00,0.00,0.50}{#1}}
\newcommand{\darkgreen}[1]
{\textcolor[rgb]{0.00,0.37,0.00}{#1}}
\newcommand{\darkred}[1]{\textcolor[rgb]{0.60,0.00,0.00}{#1}}
\newcommand{\brown}[1]{\textcolor[rgb]{0.50,0.30,0.00}{#1}}
\newcommand{\purple}[1]{\textcolor[rgb]{0.50,0.00,0.50}{#1}}

%%引用文献
\usepackage{bibentry}
\usepackage{natbib}

%%配置文章基本信息
\title{}
\author{}
\date{}


\begin{document}
\maketitle

%%摘要
\begin{abstract}
\end{abstract}

%%生成目录
\tableofcontents

\section{文本中文字效果}
\text{文本效果}
\textbf{文本效果}
{\em 文本效果}
\underline{文本效果}
\red{文本效果}
\colorbox{yellow}{文本效果}
\colorbox{yellow}{\red{文本效果么}}

\section{插入图片}

\subsection{插入单张图片}
\begin{figure}[H]
	\centering
	\includegraphics[width=0.6\textwidth]{file name}
	\caption{picture name}
	\label{my label}%%后文用\ref{my label}引用
\end{figure}

\subsection{插入过宽的单张图片}
\begin{figure}[H]
	\centering
	\includegraphics[width=0.6\textwidth][center]{file name}
	\caption{picture name}
	\label{my label}%%后文用\ref{my label}引用
\end{figure}

\subsection{插入两张图片并且并排}
\begin{figure}[H]
	\centering
	\begin{minipage}{0.48\textwidth}
		\centering
		\includegraphics[width=0.8\textwidth]{file name}
		\caption{picture name}
	\end{minipage}
	\begin{minipage}{0.48\textwidth}
		\centering
		\includegraphics[width=0.8\textwidth]{file name}
		\caption{picture name}
	\end{minipage}
\end{figure}

\subsection{插入三张图片并且并排}
\begin{figure}[H]
	\centering
	\begin{minipage}{0.32\textwidth}
		\centering
		\includegraphics[width=0.8\textwidth]{file name}
		\caption{picture name}
	\end{minipage}
	\begin{minipage}{0.32\textwidth}
		\centering
		\includegraphics[width=0.8\textwidth]{file name}
		\caption{picture name}
	\end{minipage}
	\begin{minipage}{0.32\textwidth}
		\centering
		\includegraphics[width=0.8\textwidth]{file name}
		\caption{picture name}
	\end{minipage}
\end{figure}

\subsection{插入包含子图的图片}
\begin{figure}[H]
	\centering
	\subfigure[caption 1]{
		\includegraphics
		[width=0.2\textwidth]
		{file name}}
	\subfigure[caption 2]{
		\includegraphics
		[width=0.2\textwidth]
		{file name}}
	\caption{}
\end{figure}


\section{表格设置}
%%注意两个环境之间加入
\caption{name}
%%表格编辑网站:https://www.tablesgenerator.com/latex_tables

\section{公式设置}
\subsection{普通公式}
\begin{equation}
	\tag{}%%可以自己设置编号,也可以按系统自己编号的来\notag表示没有编号
	\label{my label}%%也可以添加引用,后文用\eqref{my label}即可,可以方便的引用公式
\end{equation}
\subsection{多个公式且控制位置}
\begin{align}
	 & \notag \\ %%双反斜杠控制变行,\notag控制编号
	 &
\end{align}
\subsection{多个公式无需控制位置}
\begin{gather}
	a=b+c\\
	c=b+d\\
	d=r+a\\
	l=m+n
\end{gather}
\subsection{同一个公式换行并将公式符号居中}
\begin{equation}
	\begin{aligned}
		a & =b+c \\
		c & =c+d
	\end{aligned}
\end{equation}


\section{参考文献设置}
\subsection{手动插入}
\begin{thebibliography}{99}
	\bibitem{ref1}
	Mark Srednicki.Quantum Field Theory.
	CAMBRIDGE UNIVERSITY PRESS, 1st edition, 2007.
	\bibitem{ref2}
	Zhe Chang and Sai Wang.Lorentz invariance violation
	and electrodynamic field in an iintrinsically
	anisotropic spacetime.
	The European Physical Journal C,
	arXiv:1204.2478, 72, Sept.2012
\end{thebibliography}

\subsection{自动插入}

%%新建一个.bib文件,在里面输入如下:
@ARTICLE{LLI,
	author = {},
	title = {},
	journal = {},
	year = {},
	volume = {},
	month = {},
	publisher = {}
}
@book{SK,
	author = {},
	title = {},
	publisher = {},
	year = {},
	edition = {},
}
%%在需要插入参考文献的区域内插入:
\bibliographystyle{unsrt}
\bibliography{sample}
%%在文章中的引用:
\cite{SK}
\nocite{LLI}
%%注意,参考了但是没有地方引用一定要选择nocite,每天文献要生成条目必须事先声明
\end{document}